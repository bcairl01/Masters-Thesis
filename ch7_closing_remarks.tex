%%%%%%%%%%%%%%%%%%%%%%%%%%%%%%%%%%%%%%%%%%%%%%%%%%%%%%%%%%%%%%%%%%%%%%%%%%%%%%%%%%
%%% Conclusion
%%%%%%%%%%%%%%%%%%%%%%%%%%%%%%%%%%%%%%%%%%%%%%%%%%%%%%%%%%%%%%%%%%%%%%%%%%%%%%%%%%
\chapter{Concluding Remarks}

This thesis has provided a comprehensive summary of the design and control of the BlueFoot quadruped platform. Namely, BlueFoot's structural configuration; associated component devices; gait and stability control; and related navigation control mechanisms have been described in detail. Results, both simulated and actual, have been provided about the performance of each control technique used in operating the BlueFoot platform.

Future work involving the design improvements upon the existing BlueFoot platform will be 

Future work involving the control of the BlueFoot platform will mainly deal with methods and supporting algorithms for traversing irregular terrain.  One such algorithm will involve the location and classification of rough terrain regions. Such an algorithm will be used to deduce whether or not an area of terrain is of high irregularity; as well as whether or not the terrain is traversable or needs to be traversed in order to reach a goal location. Then, the associated area will be mapped and the surface reconstruction mechanisms details in this thesis will be employed. Finally, adaptive footstep planning over the terrain will be carried out. As opposed to planning over a static terrain map, it is desired that BlueFoot continuously maps and re-plans a path for navigation over arbitrary terrain. Although adaptive planning is not always necessary, there are many environmental situations wherein an adaptive terrain traversal approach is required. For instance, one such situation occurs when the robot can only partially perceive the terrain ahead of it from its current vantage point. This would occur if the terrain being mapped was taller than the effective height of the LIDAR laser head, and the sensor could not be articulated in such a way that the entirety of an immediate terrain patch was viewable. In this case, current implementations for surface reconstruction utilize approximations for terrain beyond that of which is immediately viewable by the robot platform, such as those described in \cite{other dudes}. Instead, it is desirable to have a routine which navigates the robot based on a continuously updated terrain map. This would require that BlueFoot executes a coordinate set of motions which allow it to articulate its LIDAR sensor will walking to map the immediate terrain. From this whole-body motion planning problem. The solution to such a problem would need to address a way of combining optimal and body-motion trajectory planning to achieve both the aforementioned subtasks in concert. Additionally, the problem of online terrain mapping would require that the platform can accurately localize itself upon the known portions of the terrain while incorporating new terrain-map data, extending this study in the realm of 3D Simultaneous Localization and Mapping.