%%%%%%%%%%%%%%%%%%%%%%%%%%%%%%%%%%%%%%%%%%%%%%%%%%%%%%%%%%%%%%%%%%%%%%%%%%%%%%%%%%
%%% Conclusion
%%%%%%%%%%%%%%%%%%%%%%%%%%%%%%%%%%%%%%%%%%%%%%%%%%%%%%%%%%%%%%%%%%%%%%%%%%%%%%%%%%
\chapter{Concluding Remarks}


	This thesis has provided a comprehensive summary of the design and control of the BlueFoot quadruped platform. Namely, BlueFoot's structural makeup; component devices; software architecture; gait and stability control; and related navigation control mechanisms have been described in detail. Results from simulated and actual trials have been provided to demonstrate the performance of routines used to control the BlueFoot platform. This final section will provide some insight about future directions for the BlueFoot project, including future directions with work regarding the design of the BlueFoot platform; and with work regarding its control and navigation.


	Future work regarding the design of the existing BlueFoot platform will focus on redesigning the structure of BlueFoot's legs. The platform currently requires servos which generate higher torque outputs, particularly at its upper joints. The torque output of BlueFoots's current joint servos have shown to be adequate for moderate locomotion tasks, but may not be powerful enough for more demanding tasks over extended periods of operation. Currently, servos which actuate the upper joints tend to overheat during operation, causing them to shut down. Joint shut down is a safety feature which is part of the servo control software. This indicates that these servos are constantly operating near the output limits, even during moderately demanding tasks. To the same end, it would also be beneficial to re-incorporate series elastic joints into the robot's design for the purpose of generating more fluid gait motions and relieving dynamic torque loading during impact. The incorporation of series elastic joints would also allow for faster gaits and extended this project toward possible future studies regarding under-actuated quadruped gait control.


	Future work involving the control of the BlueFoot platform will mainly focus on research into algorithms for optimal motion planning and traversal over irregular terrain, as well as optimal whole-body motion planning towards sensor articulation and dynamic gait stability applications. In regards to rough-terrain planning, supporting algorithms must also be developed for the classification of rough terrain regions. Such routines will be necessary for deducing whether or not an area of terrain is of high irregularity; as well as whether or not the terrain is traversable or needs to be traversed in order to reach a goal location. The associated area will be mapped and represented using the surface reconstruction mechanisms detailed in this thesis. Finally, an adaptive footstep and body-placement planner will be used to navigate the robot over rough terrain. 


	The ultimate goal of this research is to design a set of control-laws for the BlueFoot platform such that the robot can simultaneously map the surrounding terrain and update its foot placement planning in real time. There are many situations wherein incremental terrain mapping and re-planning is necessary, especially in situations were information about upcoming terrain is incomplete. Specifically, one such situation arises when the terrain is dynamic, such as when the terrain can be modified when the robot makes contact with it. Another obvious situation arises when the robot can only partially perceive upcoming terrain from its current vantage point. This could occur if the terrain being mapped was taller than the effective viewing height of the vision sensor being utilized to generate the associated terrain map.

	In the event of missing terrain data, current implementations utilize approximations the for terrain beyond the portions which are immediately perceivable, such as those described in \cite{Kolter2009}. Along with improving these types of techniques, it is also important develop a set of techniques which can be used to navigate the robot based on a continuously updated terrain map. In BlueFoot's case, this would mean that the robot must execute a coordinated set of motions which allow it to articulate its LIDAR sensor, for the purpose of effective terrain mapping, while also walking over said terrain. From this, a complicated whole-body motion planning problem arises. The solution to such a problem would need to address a method for generating optimal body-motion trajectories, which simultaneously achieve a stable robot configuration during gaiting, while also generating effective sweeping motions for terrain scanning with the body-mounted LIDAR. This problem also demands an accurate localization of the robot. For this, methods for simultaneous localization and mapping in 3D will have to be explored. Additionally, terrain features could be used to localize the robot based on its kinematic configuration. In such a localization scheme, robot orientation, foot contact, and joint position data could be used to estimate the robot's kinematic pose from sets of terrain features representing possible foot contact points.

	Additional problems arise from the root rough-terrain planning problem, including an assignment of cost to the act of actually mapping a region of terrain. In this regard, one must consider the time and effort (energy demands, computational effort, storage demands etc.) required for completing a terrain map of a particular region. In BlueFoot's case, a significant amount of control effort is spent during terrain scanning. If we consider a routine similar to BlueFoot's current terrain-scanning routine, in which the robot comes to a complete stop before sweeping its body to generate varying scan levels, then the cost of generating information about the upcoming terrain  manifests in both the time and physical energy spent collecting terrain information. In a more elegant control scheme, where the robot simultaneously moves and maps its terrain, this cost manifests as extra energy exerted in the act of articulating the robot's body. From these considerations arises a question about how a robot agent can somehow predict the cost of mapping an area of terrain, from a smaller set of terrain features, before exerting the full required effort for acquiring a more detailed representation of a particular region. The robot could also learn to predict which terrains have a higher potential for successful traversal. %Navigation control could then be achieved as a fusion between optimal robot body planning in concert with optimal deduction about the path to travel as a function of the effort necessary to overcome its associated terrain.