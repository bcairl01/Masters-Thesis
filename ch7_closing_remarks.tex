%%%%%%%%%%%%%%%%%%%%%%%%%%%%%%%%%%%%%%%%%%%%%%%%%%%%%%%%%%%%%%%%%%%%%%%%%%%%%%%%%%
%%% Conclusion
%%%%%%%%%%%%%%%%%%%%%%%%%%%%%%%%%%%%%%%%%%%%%%%%%%%%%%%%%%%%%%%%%%%%%%%%%%%%%%%%%%
\chapter{Concluding Remarks}
	\label{ch::conclusion}


	This thesis has provided a comprehensive summary of the design and control of the BlueFoot quadruped platform. Namely, BlueFoot's structural makeup; component devices; software architecture; gait and stability control; and related navigation control mechanisms have been described in detail. Results from simulated and actual trials have been provided to demonstrate the performance of strategies used to control the BlueFoot platform. This final section will provide some insight about future directions for the BlueFoot project regarding the design of the BlueFoot platform; and work related to its control and navigation.


	Future work regarding the design of the existing BlueFoot platform will focus on redesigning the structure of BlueFoot's legs. After extensive real-world experimentation, it has become evident that the platform requires servos which generate higher torque outputs, particularly at its upper joints. BlueFoots's current servo outfit has shown to be adequate for moderate locomotion tasks but may not be powerful enough for more demanding tasks over extended periods of operation. Currently, servos which actuate the upper joints tend to overheat during operation, causing them to power-off. Joint shut-down is a safety feature which is part of the smart-servo control software. This indicates that these servos may be operating near their respective output limits even during moderately demanding tasks. To the same end, it would also be beneficial to re-incorporate series elastic joints into  BlueFoot's leg design for the purpose of relieving dynamic torque loading during impact, as well as to facilitate faster, more fluid-looking gaiting motions.


	Future work involving the control of the BlueFoot platform will mainly focus on research into algorithms for optimal motion planning and traversal over irregular terrain. This will include optimal whole-body motion planning towards sensor articulation and dynamic gait stability applications. In regards to rough-terrain planning, supporting algorithms must also be developed for the classification of rough terrain regions. Such routines will be necessary for deducing whether or not an area of terrain is of high irregularity; as well as whether or not the terrain is traversable. Finally, an adaptive footstep and body-placement planner will be used to navigate the robot over rough terrain. 


	The ultimate goal of this research is to design a set of control-laws for the BlueFoot platform such that the robot can simultaneously map the surrounding terrain and update its foot placement/body motion planning in real time. There are many situations wherein incremental terrain mapping and re-planning is necessary, especially when information about upcoming terrain is incomplete. Specifically, one such situation arises when the terrain is dynamic and can be modified when the robot makes contact with it. Another obvious situation arises when the robot can only partially perceive upcoming terrain from its current vantage point.

	To compensate for missing terrain data, current implementations utilize approximations the for terrain beyond the portions which are immediately perceivable, such as those described in \cite{Kolter2009}. Along with improving these terrain estimation techniques it is also important develop a set of focused strategies which motivate robot motion planning based on the completeness of the robot's current terrain model. In BlueFoot's case, this would mean that the robot would need to plan a coordinated set of motions which allow it to articulate its LIDAR sensor, for the purpose of effective terrain mapping, while also planning its walking motion over said terrain. From this, a complicated whole-body motion planning problem arises. The solution to such a problem would need to address a method for generating optimal body-motion trajectories which simultaneously achieve a stable robot configuration during gaiting while also generating effective trunk-pitching motions for terrain scanning with the body-mounted LIDAR. This problem also demands an accurate localization of the robot for generating accurate point-cloud transformations. %For this, methods for simultaneous localization and mapping in 3D will have to be explored. Additionally, terrain features could be used to localize the robot based on its kinematic configuration. In such a localization scheme, robot orientation, foot contact, and joint position data could be used to estimate the robot's kinematic pose from sets of terrain features representing possible foot contact points.

	To overcome the aforementioned problem, the ``cost" associated with the act of actually mapping a region of terrain must be considered. This cost can manifest as the time and effort (energy demands, computational effort, storage demands etc.) required for completing a terrain map of a particular region. In BlueFoot's case, a significant amount of control effort is spent during terrain scanning. If we consider a routine similar to BlueFoot's current terrain-scanning routine, in which the robot comes to a complete stop before sweeping its body to generate varying scan levels, then the cost of generating information about the upcoming terrain  manifests in both the time and physical energy spent collecting terrain information. To address this, the robot agent must somehow predict the cost of mapping an area of terrain, from a smaller set of terrain features, before exerting the full required effort for acquiring a more detailed representation of a particular region. As such, the robot could also learn to predict which terrains have a higher potential for successful traversal. %Navigation control could then be achieved as a fusion between optimal robot body planning in concert with optimal deduction about the path to travel as a function of the effort necessary to overcome its associated terrain.