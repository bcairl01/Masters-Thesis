%%%%%%%%%%%%%%%%%%%%%%%%%%%%%%%%%%%%%%%%%%%%%%%%%%%%%%%%%%%%%%%%%%%%%%%%%%%%%%%%%%
%%% Conclusion
%%%%%%%%%%%%%%%%%%%%%%%%%%%%%%%%%%%%%%%%%%%%%%%%%%%%%%%%%%%%%%%%%%%%%%%%%%%%%%%%%%
\chapter{Concluding Remarks}

This thesis has provided a comprehensive summary of the design and control of the BlueFoot quadruped platform. Namely, BlueFoot's structural configuration; associated component devices; gait and stability control; and related navigation control mechanisms have been described in detail. Results from simulated and actual trials have been provided to demonstrate the performance used to control the BlueFoot platform.

Future work involving the design improvement of the existing BlueFoot platform will focus on redesigning the structure of BlueFoot's legs. Namely, it would be beneficial outfit the platform which servos that generate higher torque outputs, particularly at the upper joints. The robot's current servo output is adequate for moderate locomotion tasks, but may not be powerful enough for more demanding tasks. Currently, servos which actuate the upper joints tend to overheat during operation, causing them to shut down. This indicates that they are constantly operating near the output limits during moderately demanding tasks. It would also be beneficial to re-incorporate series elastic joints into the robot's design for the purpose of generating more \emph{organic} gait motions. The incorporation of series elastic joints would also allow for faster gaits and extended this project toward possible future studies dealing with under-actuated control in quadruped gaiting.

Future work involving the control of the BlueFoot platform will involve research into algorithms for motion planning and traversal over irregular terrain.  One such algorithm will involve the location and classification of rough terrain regions. This routine will be necessary for deducing whether or not an area of terrain is of high irregularity; as well as whether or not the terrain is traversable or needs to be traversed in order to reach a goal location. The associated area will be mapped and the surface reconstruction mechanisms details in this thesis will be employed to represent the region as a collection of terrain features. Finally, an adaptive footstep planner will be used to move the robot over rough terrain. 

The ultimate goal of this research is to have BlueFoot incrementally map and update its foot placement and planned-path for navigation in real time. Although adaptive planning is not always necessary, there are many situations wherein incremental terrain mapping and re-planning is necessary. For instance, one such situation occurs when the robot can only partially perceive upcoming terrain from its current vantage point. This would occur if the terrain being mapped was taller than the effective height of the LIDAR laser head, and the sensor could not be articulated in such a way that the entirety of an immediate terrain patch was viewable. 

In the event of missing terrain data, current implementations utilize approximations for terrain beyond that of which is immediately viewable by the robot platform, such as that described in \cite{Kolter2009}. Instead, it is desirable to have a routine which navigates the robot based on a continuously updated terrain map. This would require that BlueFoot executes a coordinated set of motions which allow it to articulate its LIDAR sensor, for the purpose of effective terrain mapping, while also walking over said terrain. From this, a complicated whole-body motion planning problem arises. The solution to such a problem would need to address a method for generating optimal body-motion trajectories which simultaneously achieve a stable robot configuration while also generating effective sweeping motions for terrain scanning. This problem also demand accurate localization of the robot from terrain features, as vision sensors would be used to generate new knowledge about the walking surface, in particular. For this, methods for simultaneous localization and mapping in 3D will have to be explored. Additionally, terrain features could be used to localize the robot based on its kinematic configuration. In such a localization scheme, robot orientation, foot contact, and joint position data will be used to estimate its kinematic pose, which would then be used to find a possible set on the surface of a known terrain element which could represent current foot contact points. For this, localization using a particle filtering scheme could be employed in which multiple estimates of the robot upon a section of terrain are maintained.