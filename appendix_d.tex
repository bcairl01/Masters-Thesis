\chapter{C++ code Implementations for Surface Reconstruction}
	\label{appendix::d}

	\begin{frame}
	
		\lstset{ %
		  backgroundcolor=\color{white},   % choose the background color; you must add \usepackage{color} or \usepackage{xcolor}
		  basicstyle=\footnotesize,        % the size of the fonts that are used for the code
		  breakatwhitespace=false,         % sets if automatic breaks should only happen at whitespace
		  breaklines=true,                 % sets automatic line breaking
		  captionpos=b,                    % sets the caption-position to bottom
		  commentstyle=\color{cyan},    % comment style
		  deletekeywords={...},            % if you want to delete keywords from the given language
		  escapeinside={\%*}{*)},          % if you want to add LaTeX within your code
		  extendedchars=true,              % lets you use non-ASCII characters; for 8-bits encodings only, does not work with UTF-8
		  frame=single,                    % adds a frame around the code
		  fontsize=10pt,
		  keepspaces=true,                 % keeps spaces in text, useful for keeping indentation of code (possibly needs columns=flexible)
		  keywordstyle=\color{blue},       % keyword style
		  language=C,                 		% the language of the code
		  otherkeywords={*,...},            % if you want to add more keywords to the set
		  numbers=left,                    % where to put the line-numbers; possible values are (none, left, right)
		  numbersep=5pt,                   % how far the line-numbers are from the code
		  numberstyle=\tiny\color{black}, % the style that is used for the line-numbers
		  rulecolor=\color{black},         % if not set, the frame-color may be changed on line-breaks within not-black text (e.g. comments (green here))
		  showspaces=false,                % show spaces everywhere adding particular underscores; it overrides 'showstringspaces'
		  showstringspaces=false,          % underline spaces within strings only
		  showtabs=false,                  % show tabs within strings adding particular underscores
		  stepnumber=2,                    % the step between two line-numbers. If it's 1, each line will be numbered
		  stringstyle=\color{red},    	 % string literal style
		  tabsize=2,                       % sets default tabsize to 2 spaces
		  title=\lstname                   % show the filename of files included with \lstinputlisting; also try caption instead of title
		}
		
		\subsubsection{Type Definitions and Inlcudes}
\begin{lstlisting}
#ifndef POINTCLOUD_EXT_BASE_HPP
#define POINTCLOUD_EXT_BASE_HPP
	#include <pcl/common/transforms.h>
	#include <pcl/kdtree/kdtree_flann.h>
	#include <pcl/kdtree/impl/kdtree_flann.hpp>
	#include <pcl/surface/impl/mls.hpp>
	#include <pcl/filters/approximate_voxel_grid.h>
	#include <pcl/filters/passthrough.h>
	#include <pcl/point_types.h>
	#include <pcl/features/normal_3d.h>
	#include <vector>
	#include <cmath>
	#include <fstream>
	#include <iostream>
	#include <Eigen/Dense>


	typedef float 						float_t;
	typedef pcl::PointXYZ               point_t;
	typedef pcl::PointCloud<point_t>    point_cloud_t;
	typedef pcl::PointNormal            npoint_t;
	typedef pcl::PointCloud<npoint_t>   npoint_cloud_t;

	#define PCX_MSG(_str)				{std::cout << "[ PCX ] : " << _str << std::endl;}
	#define PCX_ASSERT(_x)
	#define PCX_WEAK_ASSERT(_x)
	#define PCX_CHECK_TOLKEN_CONFIRM(str0) 			{PCX_MSG("Got -- "<<str0);}
	#define PCX_CHECK_TOLKEN_GET_VAR(fs,str0,str1) {fs>>str0;PCX_CHECK_TOLKEN_CONFIRM(str0);}if(str0!=str1){PCX_MSG("Config file parsing error!"); abort();} else


#endif
\end{lstlisting}

\subsubsection{Header}
\begin{lstlisting}

#ifndef POINTCLOUD_EXT_SURFACE_HPP
#define POINTCLOUD_EXT_SURFACE_HPP 1

	
	#include "base.hpp"
	namespace pcx
	{
		namespace surface
		{


			/// @brief 	Point-cloud downsampling wrapper using 'ApproximateVoxelGrid' class
			///	@param 	[in]	src_points 		input point-cloud
			///	@param 	[out]	dst_points 		output point-cloud
			/// @param 	[in]	radius			down-sampling grid size (cube)
			/// @return none
			template< typename pointTy, typename normalTy >
				void downsample(
					typename pcl::PointCloud<pointTy>::Ptr&  	src_points,
					typename pcl::PointCloud<pointTy>::Ptr&  	dst_points,
					const float_t 								radius
				){
					pcl::ApproximateVoxelGrid<pointTy> sor;
					sor.setInputCloud(src_points);
					sor.setLeafSize(radius, radius, radius);
					sor.filter(*dst_points);
				}
			template< typename pointTy, typename normalTy >
				void downsample(
					typename pcl::PointCloud<pointTy>::Ptr&  	src_points,
					typename pcl::PointCloud<pointTy>&  		dst_points,
					const float_t 								radius
				){
					pcl::ApproximateVoxelGrid<pointTy> sor;
					sor.setInputCloud(src_points);
					sor.setLeafSize(radius, radius, radius);
					sor.filter(dst_points);
				}



			/// @brief Surface-Nomal estimation wrapper with an available 'searchRadius' parameter.
			///	@param 	[in]	src_points 		source point cloud
			///	@param 	[out]	dst_normals 	destination point-normal cloud
			/// @param 	[in]	searchRadius	local search radius for normal estimation
			/// @return none
			template< typename pointTy, typename normalTy >
				void estimate_normals(
					typename pcl::PointCloud< pointTy>::Ptr&  	src_points,
					typename pcl::PointCloud<normalTy>::Ptr&  	dst_normals,
					const float_t 								searchRadius = 0.03f
				){
					// Create a new kdTree for the given source entry
					typename pcl::search::KdTree<pointTy>::Ptr tree (new pcl::search::KdTree<pointTy>());

					// Setup the normals-estimation engine
					typename pcl::NormalEstimation<pointTy, normalTy> normals_engine;

					normals_engine.setInputCloud (src_points);
					normals_engine.setSearchMethod (tree);
					normals_engine.setRadiusSearch (searchRadius);

					// Compute the features
					normals_engine.compute (*dst_normals);
				}


			/// @brief Surface-Nomal estimation wrapper with an available 'searchRadius' parameter.
			///	@param 	[in]	src_normals 	source point-normal cloud
			///	@param 	[out]	dst_indices 	destination vector of indices corresponding to alligned vectors in 'src_normals'
			/// @param 	[in]	allign_to		vector to match (allign) against
			/// @param 	[in]	allign_conf_min	minimum allignement confidence [0.0,1.0]
			/// @return none
			template< typename pointTy, typename normalTy, typename indexTy >
				void find_alligned_normals(
					typename pcl::PointCloud<normalTy>::Ptr&  	src_normals,
					typename std::vector<indexTy>&				dst_indices,
					Eigen::Vector3f&          					allign_to,
					float           							allign_conf_min = 0.75f
				){
					PCX_ASSERT(allign_conf_min>=0);
					PCX_ASSERT(allign_conf_min<=1);

					for( indexTy idx = 0; idx < src_normals->size(); ++idx )
					{
						float allignment = (*src_normals)[idx].getNormalVector3fMap().dot(allign_to);
						if( allignment > allign_conf_min )
							dst_indices.push_back(idx);
					}
				}


			/// @brief 	Create a height-map (graded occupancy grid) from a 3D point cloud
			///	@param 	[in]	src_points 		source point-cloud
			///	@param 	[out]	dst_heightfield WxD hieghtfield matrix (occ. grid)
			/// @param 	[in]	true_width		real width (m) of the area
			/// @param 	[in]	true_depth		real width (m) of the area
			/// @param 	[in] 	offset 			hieghtfield matrix height-entry offset
			/// @return none
			template< typename pointTy, size_t sW = 100, size_t sD = 100 > 
				void to_cvheightmap(
					typename pcl::PointCloud< pointTy>::Ptr&  	src_points,
					typename Eigen::Matrix<float_t,sW,sD>&		dst_heightfield,	
					const float_t 								true_width,
					const float_t 								true_depth,
					const float_t 								offset 		= 0.f
				){
					float_t Wps 	= true_width / ( (float_t) sW );
					float_t Dps 	= true_depth / ( (float_t) sD );

					PCX_ASSERT(sW >= 10);
					PCX_ASSERT(sD >= 10);
					PCX_WEAK_ASSERT(Wps>=0.f);
					PCX_WEAK_ASSERT(Dps>=0.f);

					/// Init the field
					dst_heightfield = Eigen::Matrix<float_t,sW,sD>::Zero();

					/// Find the highest grid occupancies
					for( size_t idx = 0UL; idx < src_points->size(); ++idx )
					{
						size_t 	Xdx 	= sW - floor( ( (*src_points)[idx].x + true_width/2.f ) / Wps ) - 1UL;
						size_t 	Jdx 	= floor( ( (*src_points)[idx].y + true_width/2.f ) / Dps );

						if( (Xdx < sW) && (Jdx < sD) )
						{	
							float_t height 	= ( (*src_points)[idx].y + offset );
							if( height >  dst_heightfield(Xdx,Jdx) )
								dst_heightfield(Xdx,Jdx) = height;
						}
					}
				}
		}
	}

	#endif
\end{lstlisting}
		
	\end{frame}