%%%%%%%%%%%%%%%%%%%%%%%%%%%%%%%%%%%%%%%%%%%%%%%%%%%%%%%%%%%%%%%%%%%%%%%%%%%%%%%%%%
%%% Control
%%%
%%% NEED TO FIX NOTATIONS
%%%
%%%%%%%%%%%%%%%%%%%%%%%%%%%%%%%%%%%%%%%%%%%%%%%%%%%%%%%%%%%%%%%%%%%%%%%%%%%%%%%%%%
\chapter{System Modeling}
	

	%Analysis and control of the BlueFoot quadruped requires the utilization of both kinematic and dynamical system models.
	
	\section{Kinematics Model}
		\label{sec::kinematics_model}

		The kinematic model of the BlueFoot platform is paramount for foot/body trajectory planning, localization, and adaptation. In particular, inverse position and velocity solutions are used to prescribe joint-space commands from particular foot trajectories prescribed in the world frame. Forward kinematic solutions are utilized to localize the position of each foot using a combination of localized trunk position/orientation and joint position feedback. This section will detail the forward and inverse kinematic models (for both position and velocity) which describe the BlueFoot platform.

		\subsection{Forward Position Kinematics}

			To formulate the kinematics model, a set of coordinate systems have been defined and are described by \ref{fig::coordinate_frames}. Note that the frame $O_{0}$ represents the world coordinate frame; and $O_{b}$ is the coordinate frame, centered at  ${p}_{b}$ attached to the platform and is always aligned with $O_{0}$. $O_{b}$ represents a body frame rigidly attached to the center of the trunk. The orientation and position of the trunk are defined by vectors of $\theta_{b}\InRe{3}$ and ${p}_{b}\InRe{3}$, which relate the frame $O_{b}$ to the world frame $O_{0}$. Coordinate frames $O^{i,0}$ are attached to the first joint of each \Ith leg.

				\begin{figure}[h!]
					\centering
					\fbox{\includegraphics[width=0.75\textwidth]{coordinate_frames.png}}
					\caption{Coordinate frame setup.}
					\label{fig::coordinate_frames}
				\end{figure}

			The \Jth joint position of each \Ith leg is represented  by the points ${p}_{i,j}$ in the frame $O_{0}$. These spatial locations are generated from a combination of the body orientation, $\theta_{b}$, and joint positions for each \Ith leg, $q_{i} = [q_{i,1}, q_{i,2}, q_{i,3}, q_{i,4}]^T$. $q_{i,1}$ represents the position of the hip-joint (joint closest to the center platform), which rotates in the direction of the transverse body plane. The joint variables $q_{i,2}, q_{i,3}$ and $q_{i,4}$ represent the lateral-hip, knee and ankle joint rotations, respectively.

			The coordinate transformation between world coordinate frame, $O_{0}$, and the zeroth Denavit-Hartenberg (DH) coordinate frame of leg \emph{i}, $O^{i,0}$ (located at the origin of joint-1), is given by
				\begin{equation}
					H_{0}^{i,0} = \left[ 
					\begin{array}{c|c}
						R_{zyx}(\theta_{b}) R_{z}(\sigma_{i})	&R_{zyx}(\theta_{b}) \nu + {p}_{b} 	\\ \hline
						0										&	1											\\
					\end{array} 
					\right]
					\label{eq::world_to_dh}
				\end{equation}
			where $\sigma_{i} \equiv \frac{\pi}{2}(i-1) + \frac{\pi}{4} $ and $\nu \equiv  R_{z}(\sigma_{i}) \beta$ with $\beta$ defining an offset from $O_{b}$ to where the first joint of each leg is attached to the body. $R_{zyx}$ represents a rotation associated with the pitch (x-axis), roll (y-axis), and yaw (z-axis) angles of the main body about the platform frame, $\theta_{b}$. $R_{z}$ represents a rotation about the (z-axis) of the frame $O_{b}$. 

			A transformation from the zeroth DH frame of the to the $(j+1)^{th}$ joint of leg \emph{i} is described, in general, by
				\begin{equation}
					H^{i,0}_{i,j} =
					\left[ 
					\begin{array}{c|c}
						R^{i,0}_{i,j} 	&	{p}^{i,0}_{i,j} 	\\ \hline
						0			&	1				\\
					\end{array} 
					\right].
				\end{equation}

			\noindent
			The kinematics of each leg are identical. Thus, the transformations $H^{i,0}_{i,j}$ are of the same form and are derived by the DH parameters given by Table \ref{tab::dh_params}. Actual values for the link lengths $a_{1-4}$ and body-offset, $\nu$, are provided in Table \ref{tab::link_lens}.


			\begin{table}[h]
				\centering
				\begin{tabularx}{150mm}{|C{0.2}|C{0.2}|C{0.2}|C{0.2}|C{0.2}|} \hline
					\textbf{Link}	&\textbf{\small{Length} : $a_i$} &	\textbf{\small{Twist} : $\alpha_i$}	&	\textbf{\small{Offset}: $d_i$}	&	\textbf{\small{Rotation} : $\theta_i$} \\ \hline \hline
					1				&	$a_{1}$		&	$\pi/2$				&	0				&	$q_{i,1}^*$			\\ \hline
					2				&	$a_{2}$		&	0					&	0				&	$q_{i,2}^*$			\\ \hline
					3				&	$a_{3}$		&	0					&	0				&	$q_{i,3}^*$			\\ \hline
					4 				&	$a_{4}$		&	0					&	0				&	$q_{i,4}^*$			\\ \hline
				\end{tabularx}
				\caption{DH parameters for all legs.}
				\label{tab::dh_params}
			\end{table}
			
			\noindent
			Using these DH parameters, the transformations $H^{i,0}_{i,1}$, $H^{i,0}_{i,2}$, $H^{i,0}_{i,3}$, and $H^{i,0}_{i,4}$ can be computed explicitly as follows:

				%% Manipulator Base to Frame 1 %%
				\begin{equation} 
					H^{i,0}_{i,1} =\left[ 
					\begin{array}{ccc|c}
						c_{1,i} 	&  		0	& 	s_{1,i}		&		c_{1,i} a_{1,i}	\\
						s_{1,i} 	&  		0	& 	-c_{1,i}	&		s_{1,i} a_{1,i}	\\
						0 			&  		1	& 	0			&		0 				\\ \hline
						0 			&  		0	& 	0			&		1 				\\
					\end{array} 
					\right]
				\end{equation}


				%% Manipulator Base to Frame 2 %%
				\begin{equation}
					H^{i,0}_{i,2} =\left[ 
					\begin{array}{ccc|c}
						c_{1,i} c_{2,i}	&  		-c_{1,i} s_{2,i}	& 		s_{1,i}		&	c_{1,i}( a_{1,i} + a_2 c_{2,i} )\\
						s_{1,i} c_{2,i}	&  		-s_{1,i} s_{2,i}	& 		-c_{1,i}	&	s_{1,i}( a_{1,i} + a_2 c_{2,i} )\\
						s_{2,i} 		&  		c_{2,i}			 	& 		0			&	a_2 s_{2,i} 					\\ \hline
						0 			&  		0			 		& 		0			&	1 								\\
					\end{array} 
					\right]
				\end{equation}


				%% Manipulator Base to Frame 3 %%
				\begin{equation}
					H^{i,0}_{i,3} =\left[ 
					\begin{array}{ccc|c}
						c_{1,i} c_{23,i}	&  		-c_{1,i} s_{23,i}	& 		s_{1,i}		&		c_{1,i}( a_{1,i} + a_2 c_{2,i} + a_3 c_{23,i} )	\\
						s_{1,i} c_{23,i}	&  		-s_{1,i} s_{23,i}	& 		-c_{1,i}	&		s_{1,i}( a_{1,i} + a_2 c_{2,i} + a_3 c_{23,i} )	\\
						s_{23,i} 		&  		c_{23,i}			& 		0			&		a_2 s_{2,i} + a_{3,i} s_{23,i}			\\ \hline
						0 			&  		0					& 		0			&		1 										\\
					\end{array} 
					\right]
				\end{equation}

				%% Manipulator Base to Frame 4 %%
				\begin{equation}
					H^{i,0}_{i,4} =\left[ 
					\begin{array}{ccc|c}
						c_{1,i} c_{234,i}	&  		-c_{1,i} s_{234,i}	& 		s_{1,i}		&		c_{1,i}( a_{1,i} + a_2 c_{2,i} + a_3 c_{23,i} + a_4 c_{234,i} )		\\
						s_{1,i} c_{234,i}	&  		-s_{1,i} s_{234,i}	& 		-c_{1,i}	&		s_{1,i}( a_{1,i} + a_2 c_{2,i} + a_3 c_{23,i} + a_4 c_{234,i} )		\\
						s_{234,i} 		&  		c_{234,i}		& 		0			&		a_2 s_{2,i} + a_3 s_{23,i} + a_4 s_{234,i}					\\ \hline
						0 			& 		0			& 		0			&		1 															\\
					\end{array} 
					\right]
				\end{equation}

			\noindent
			The position of joint 1 of leg \emph{i} in $O_{0}$, ${p}_{i,1}$ may now be computed with respect to frame $O_{0}$ by:
				\begin{equation}
					{p}_{i,1} \equiv E_{p,1} H^{0}_{i,0} E_{p,2}.
				\end{equation}
			where
				\begin{eqnarray}
					E_{p,1} = [I_{3\times3},0_{3\times1}]	\nonumber 	\\
					E_{p,2} = [0_{1\times3},1]^T.			\nonumber 	
				\end{eqnarray}
			The position of joints 2-4 of leg \emph{i}, may now be computed with respect to frame $O_{0}$ by:
				\begin{equation}
					{p}_{i,j} \equiv E_{p,1} H^{0}_{i,0} {H}^{i,0}_{i,(j-1)} E_{p,2}. \Sep \forall \Sep j\in{2,3,4}
				\end{equation}
			Finally, the position of the end-effector (foot) of \Ith leg. ${p}_{i,e}$ with respect to frame $O_{0}$, is obtained as follows:
				\begin{equation}
					{p}_{i,e} \equiv E_{p,1} H^{0}_{i,0} {H}^{i,0}_{i,4} E_{p,2}.
				\end{equation}


		




		\subsection{Inverse Position Kinematics}

			A foot configuration is specified by its coordinates ${p}_{i,e}$ and an ankle orientation, $\gamma_{i}$,  which represents a rotation about the axis of rotation of the second joint (lateral hip). Given a desired platform configuration, $\{ {p}_{b}, \theta_{b} \}$,  and desired \Ith foot configuration,  $\{ {p}_{i,e} , \gamma_{i} \}$, the inverse kinematics solution for each \Ith leg, ${q}_{i}$, is derived to be:

				\begin{eqnarray}
					q_{i,1} &=& \cos(i\pi) \wrap{ \frac{\pi}{4} - \psi_{i} } \nonumber\\
					q_{i,2} &=&	\tan^{-1} \wrap{\frac{\zeta_{i,z}}{\sqrt{\zeta_{i,x}^2+\zeta_{i,y}^2}}} \mp \cos^{-1}\wrap{\frac{a_{3}^2-a_{2}^2-\norm{\zeta_{i}}^2}{2 a_{2} \norm{\zeta_{i}} }} \pm \pi 	\nonumber\\
					q_{i,3} &=&	\mp \cos^{-1}\frac{\norm{\zeta}^2-a_{2}^2-a_{3}^2}{2 a_{2} a_{3}} \nonumber\\
					q_{i,4} &=&	\gamma_{i} - q_{i,2} - q_{i,3}	
				\end{eqnarray}
				%
				%
				where
				%
				%
				\begin{eqnarray}
					{p}_{i,e}^{i,0} &=&
					E_{p,1} 
					\wrap{ {H}_{i,k}^{0} } ^{-1}
					\left[
						\begin{array}{c}
							{p}_{i,e} 		\\
							1 				\\ 	
						\end{array}
					\right]	\nonumber\\																						\nonumber\\
					\psi_{i} 	&\equiv&	\tan^{-1}\wrap{ ({p}_{i,e}^{i,0})_{2}/({p}_{i,e}^{i,0})_{1} }												\nonumber\\
					\zeta_{i,x} &\equiv& 	({p}_{i,e}^{i,0})_{2} \sin(\psi_{i}) + ({p}_{i,e}^{i,0})_{1} \cos(\psi_{i}) - a_{4} \cos(\gamma_{i}) - a_{1} 						\nonumber\\
					\zeta_{i,y} &\equiv& 	({p}_{i,e}^{i,0})_{2} \cos(\psi_{i}) - ({p}_{i,e}^{i,0})_{1} \sin(\psi_{i}) 											\nonumber\\
					\zeta_{i,z}	&\equiv&  	({p}_{i,e}^{i,0})_{3} - a_{4} \sin(\gamma_{i}) .
				\end{eqnarray}
			Here, ${p}_{i,e}^{i,0}$ represents the position of each \Ith foot with respect to the zeroth DH frame of each \Ith leg, $O_{i,1}$; and $({p}_{i,e}^{i,0})_{k} \Sep k\in\{1,2,3\}$ represents the \Kth element of ${p}_{i,e}^{i,0}$.

			It is important to note that the ankle specification,  $\gamma_{i}$, adds extra constraints on the system kinematics and, thus, reduces the number of inverse kinematics solutions per desired foot position to two.



		\subsection{Velocity Kinematics}

			Using the forward position kinematic model detailed in Section \ref{sec::kinematics_model}, the velocity kinematics of each \Ith leg are derived to yield a manipulator Jacobian $J^{i,0}_{i,e} \InRe{6\times4}$, for each \Ith leg. This matrix provide a transformation from the local workspace of each leg to the joint space of said leg, \IE $ \dot{x}^{i,0}_{i,e} = J^{i,0}_{i,e}  \dot{q}_{i}$. Here, $\dot{x}^{i,0}_{i,e} \InRe{6}$ defines a stacked vector of translational and rotational velocities, $\dot{p}_{i,e}^{i,0} \InRe{3}$ and $\dot{\theta}_{i,e}^{i,0}\InRe{3}$, respectively, of each \Ith foot with respect to frame $O_{i,0}$. The matrix $J^{i,0}_{i,e}$ is defined explicitly in Appendix [B] using the previously defined notations.

			Assuming the translational and rotational velocity of the trunk, $\dot{p}_{b}$ and $\dot{\theta}_{b}$, respectively; and the translational and rotational of each \Ith foot, $\dot{p}_{i,e} \InRe{3}$ and $\dot{\theta}_{i,e} \InRe{3}$, respectively, are known, the translational velocity of each \Ith foot can be written with respect to frame $O_{i,0}$ by:
				\begin{equation}
					\dot{p}_{i,e}^{i,0} \equiv \wrap{R^{0}_{i,0}}^{T} \wrap{ \dot{p}_{i,e} - \dot{p}_{b} - \Skew{ \dot{\theta}_{b} } \wrap{ {p}_{i,e} - {p}_{b} - R^{0}_{i,0} \vec{o}_{\nu} } }
					\label{eq::local_foot_vel_from_world}
				\end{equation}
			where $\vec{o}_{\nu} = [\nu,0,0]^{T}$, $R^{0}_{i,0}$ is the rotation-matrix component of the transformation $H^{0}_{i,0}$ defined in \ref{eq::world_to_dh}, and $\Skew{*}$ is the standard skew-symmetric matrix operator, which takes a $3\times1$ vector as an argument. The corresponding rotational velocity of each \Ith foot  can be computed with respect to $O_{i,0}$ by:
				\begin{equation}
					\wrap{ R^{0}_{i,0} }^{T} \Skew{ \dot{\theta}_{i,e} - \dot{\theta}_{b} } R^{0}_{i,0}  = \Skew{ \dot{\theta}^{i,0}_{i,e} } = S^{i,0}_{i,e}
					\label{eq::local_foot_omega_from_world}
				\end{equation}
			where the rotational velocity vector $\dot{\theta}^{i,0}_{i,e}$ is recovered by:
				\begin{equation}
					\dot{\theta}^{i,0}_{i,e} \equiv \sbrack{ 
						-\wrap{S^{i,0}_{i,e}}_{1,2},
						 \wrap{S^{i,0}_{i,e}}_{1,3},
						-\wrap{S^{i,0}_{i,e}}_{2,3}
					}^{T}.
					\label{eq::local_foot_omega_extraction}
				\end{equation}
			Here, $\wrap{S^{i,0}_{i,e}}_{j,k}$ selects the element in the \Jth-row and \Kth-collumn of the matrix $S^{i,0}_{i,e}$. 

			Using $J^{i,0}_{i,e}$, the joint velocities, $\dot{q}_{i}$, can now be computed from the world-frame trajectories $\dot{x}_{i,e}$. Since each of BlueFoot's of legs had 4 degrees of freedom, $J^{i,0}_{i,e}$ is rank deficient and $\dot{q}_{i}$ cannot be obtained exactly by means of direct inversion. Instead, $\dot{q}_{i}$ can be approximated, in a least-squares sense, from $\dot{x}^{i,0}_{i,e}$ using the Penrose-Moore psuedo-inverse $\wrap{J^{i,0}_{i,e}}^{\dagger}$, defined as follows:
				\begin{equation}
					\dot{q}_{i} \approx \wrap{J^{i,0}_{i,e}}^{\dagger} \dot{x}^{i,0}_{i,e} \equiv \sbrack{ \wrap{J^{i,0}_{i,e}}^{T} J^{i,0}_{i,e} }^{-1} \wrap{J^{i,0}_{i,e}}^{T} \dot{x}^{i,0}_{i,e}
					\label{eq::jabobian_pseudo_inverse}
				\end{equation}

			The collection of relationships \ref{eq::local_foot_vel_from_world}, \ref{eq::local_foot_omega_from_world}, \ref{eq::local_foot_omega_extraction}, and \ref{eq::jabobian_pseudo_inverse} are are particularly useful for obtaining reference joint-velocities from reference foot trajectories, $\dot{x}^{r}_{i,e}$, planned in the frame $O_{0}$, given a known trunk pose, $\{ p_{b}, \theta_{b} \}.$





	\section{Dynamical Model}
	

		\subsection{System State Vector and General-Form Dynamics}
			\label{sec::state_vector}
		
			To fully defined the state of the BlueFoot platform, we consider a general, free-floating (6 translational/rotational degrees of freedom), four legged robotic system with four degrees of freedom per leg. This system is fully described by the state vector $z \equiv [\eta^{T}, \dot{\eta}^{T}]^{T} \RealVec{44}$ and its dynamics are:
				\begin{equation}
					M(\eta)\ddot{\eta} + C(\eta,\dot{\eta})\dot{\eta} + G(\eta) + \Delta{H} = \tau + J^T(\eta) f_{ext} %
					\label{eq::normal_form_dynamics}
				\end{equation}
			where $M(\eta)$, $C(\eta,\dot{\eta}$), $G(\eta)$ and $J(\eta)$ represent the system mass matrix, Coriolis matrix, gravity matrix and Jacobian, respectively \cite{Wieber2006}. $\Delta{H}$ has been included as a lump term to account for dynamical uncertainties, such as friction or unmodeled coupling effects. Additionally, $f_{ext} = [ f_{1,ext}^{T},f_{2,ext}^{T},f_{3,ext}^{T},f_{4,ext}^{T}]^{T} \RealVec{24}$ represents a stacked vector of force-wrenches, $f_{i,ext} \RealVec{6}$, applied to the system through each \Ith foot. The state vector, $\eta$, can be partitioned as follows: $\eta = [ {p}_{b}^{T}, \theta_{b}^{T}, q^{T} ]^{T}$ with ${p}_{b} \RealVec{3}$ and $\theta_{b} \RealVec{3}$ representing the position and orientation, respectively, of the quadruped's trunk in an arbitrarily placed world coordinate frame, and $q \RealVec{16}$ is a vector of joint variables, $m$ of which are contributed by each leg. $\tau \RealVec{22}$ represents a vector of generalized torque inputs and takes the form $\tau = [ 0_{1x6}, \tau_{q}^{T} ]^{T}$ where $\tau_{q}$ represents a set of torque inputs to each joint. It is important to note that the states we are most interested in controlling, ${p}_{b}$ and $\theta_{b}$, are not directly actuated, and must be controlled via composite leg joint motions.		

			The dynamics in \ref{eq::normal_form_dynamics} can be realized in compact, state-space form by:
				\begin{equation}
					\begin{split}
					\dot{z}_{1} 				&= z_{2} \\
					\dot{z}_{2} 				&= M^{-1}(z_{1})(\tau + \Phi(z_{1},z_{2},f_{ext})) \\
					\Phi(z_{1},z_{2},f_{ext}) 	&= J^T(z_{1}) f_{ext} - C(z_{1},z_{2})z_{2} - G(z_{1}) - \Delta{H}
					\end{split}
					\label{eq::state_space_dynamics}
				\end{equation}
			where $z_{1}=\eta$ and $z_{2}=\dot{\eta}$. The notation $\Phi(z_{1},z_{2},f_{ext})$ is introduced for convenience as a composite dynamical term. This term will be referred to, simply, as $\Phi$ in the sections that follow.

			The above system dynamics can also be considered in an approximate, discrete-time (first-order) form as follows:
				\begin{equation}
					\begin{split}
					{z}_{1,k+1} &= {z}_{1,k} + ( {e}_{1,k}^{\Delta_{s}} + {z}_{2,k} )\Delta_{s} \\
					{z}_{2,k+1} &= {z}_{2,k} + M^{-1}_{1,k}( {e}_{2,k}^{\Delta_{s}} + \tau_{k} + \Phi_{k}) \Delta_{s} \\
					t 			&= \Delta_{s} k
					\end{split}
					\label{eq::sampled_dynamics}
				\end{equation}
			where $M_{1,k} = M(z_{1,k})$ and $\Delta_{s} \equiv (f_{s})^{-1}$ with $f_{s}$ defining a uniform sampling frequency in Hz. The terms ${e}_{1,k}^{\Delta_{s}}$ and ${e}_{2,k}^{\Delta_{s}}$ are used to explicitly account for system discretization errors, which vary with respect to the step-size, $\Delta_{s}$. This representation of the system dynamics will be utilized in the chapters that follow for the use in a trunk-leveling controller.


		\subsection{Joint-Servo Dynamics}
			\label{sec::joint_dynamics}

			The motor dynamics driving each joint need to be considered for use in control design since the input to BlueFoot's servo motors at each joint is a reference position command. In model-based control schemes to follow, a simple model of the motor dynamics will be utilized. Moreover, servos are considered as simple torque generators of the following form:
				\begin{equation}
					\tau_{q} = k_{s}(q^{r}-q)
					\label{eq::servo_control_dynamics}
				\end{equation}
			where $k_{s}>0$ is a constant, scalar gain and $q^{r}$ is a joint position reference. The servos we are utilizing to drive the leg joints of the BlueFoot quadruped have high-gain position feedback which allows us to model the motors, simply, as a static block which transform reference trajectories to torque outputs. All of these servos are identical, and thus have identical gains. One could instead consider the full motor dynamics for computing reference positions given a desired torque. The simple model stated above was adequate for achieving desired results with all proposed control schemes which use this torque-generator model.


		\subsection{Single Leg Dynamics}
			\label{sec::leg_dynamics}

			The dynamics of each independent leg (with each base-frame fixed) can be derived analytically and in closed form. These dynamics have been included in Appending [B], for the sake of completeness, using the previously defined notations.


