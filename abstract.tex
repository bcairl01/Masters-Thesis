{\msabstract{Dr. Farshad Khorrami}}
{
	This thesis presents the development and control of a small-scale quadruped robot platform with 16 actuated degrees-of-freedom, named ``BlueFoot." The BlueFoot platform has been developed for the purpose of studying multi-terrain navigation and gait control in concert with full-body actuation, which may be used for reorienting payloads (\EG laser distance sensor and vision-sensor peripherals). This thesis will detail the design of the BlueFoot platform and its hardware sub-systems; an in-depth analysis of the system's kinematic model and robot dynamics; core Central-Pattern Generator (CPG) based gaiting algorithms introducing reflexive, feedback-driven mechanisms; and a unique foot placement and Zero-Moment Point (ZMP) posture controller based on a virtual force model and a posture feedback loop utilizing inertial measurements. %Lastly, this paper will include results from an object tracking task performed by the BlueFoot platform which utilizes a full-body actuation controller to track a moving target object through a combination of body pitch and roll adjustments, and omni-directional gaiting. 

	In addition, this thesis offers a method for attaining constant orientation of the trunk of a multi-legged (here a quadruped) robot in the presence of disturbances due to feet impact with the ground. This is significant when payloads (such as cameras, optical systems, armaments) are carried by the robot.  The trunk is stabilized by the utilization of an on-line learning method to actively correct the open-loop gait generated by a CPG or a limit-cycle method. The learning method is based on a Nonlinear Autoregressive Neural Network with Exogenous inputs (NARX-NN)-- a recurrent neural network architecture typically utilized for modeling nonlinear difference systems. A supervised learning approach is used to train the NARX-NN. 
	%The input to the neural network includes states of the robot legs, trunk attitude and attitude rates, and feet contact forces. The neural network is used to generate the total torque imparted on the robot. This approach allows on-line learning of the internal forces and disturbances due to various effects to be estimated/learned with the neural network for implementation in an inverse dynamics/computed torque controller. The controller is utilized to achieve a stable trunk (\IE a constant orientation of the trunk). 
	The efficacy of the proposed approach is shown in detailed simulation studies of a quadruped robot. 

	Lastly, this thesis will present several algorithms related to navigation control, terrain modeling, and rough-terrain gait planning. In particular, algorithms for surface reconstruction and foothold planning over uneven terrain will be integral components for future developments related to the BlueFoot project. Results from simulations and actual robot trials will be presented to demonstrate the performance of these control strategies. 
}
{\endmsabstract}